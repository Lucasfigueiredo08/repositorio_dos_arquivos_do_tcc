\chapter{Fundamentação Teórica}

\section{Heroku}

O Heroku é uma plataforma como serviços (PaaS) utilizada para facilitar o deploy de aplicações backend, testes em produção ou hospedagem de sites ou APIs. Bastante utilizada, o Heroku tem integrações com o GitHub, facilitando ainda mais a vida de seus usuários, suas atualizações de código são automaticamente replicadas para a aplicação na medida que o seu repositório é alterado no GitHub.

\section{Android Studio}

\section{MVC}

\section{Laravel}

\section{Firebase}

Firebase é uma plataforma digital desenvolvida pela google utilizada para facilitar o desenvolvimento de aplicativos web ou móveis, de uma forma efetiva, rápida e simples. Suas funcionalidades principais são Firebase Authentication, que fica responsável por toda da parte de autenticação dos aplicativos, Cloud Messaging, responsável por toda parte de notificações da aplicação, podendo notificar várias plataformas, e o Realtime Messaging, onde é utilizado para mensagens instantâneas, a exemplo, o Whatsapp.

\section{Java}

Java é uma linguagem de programação orientada a objetos criada em 1990 pela empresa Sun Microsystems e hoje pertence a Oracle. O java é uma lingaguem multiplataforma, podendo ser utilizando tanto em navegadores quanto em dispositivos para smartphones, Windows, linux e outros perífericos isso porque o código Java não é compilado para uma linguagem de máquina e sim para uma linguagem intermediária chamada bytecode que é interpretada e executava pela máquina virtual (JVM) Java.