%\chapter{Fundamentação Teórica}

\begin{comment}
\chapter{Ferramentas}
\section{Java}

Java é uma linguagem de programação orientada a objetos criada em 1990 pela empresa Sun Microsystems e hoje pertence a Oracle. O java é uma lingaguem multiplataforma, podendo ser utilizando tanto em navegadores quanto em dispositivos para smartphones, Windows, linux e outros perífericos isso porque o código Java não é compilado para uma linguagem de máquina e sim para uma linguagem intermediária chamada bytecode que é interpretada e executava pela máquina virtual (JVM) Java.

\section{Android Studio}

O Android Studio é chamado de Ambiente de Desenvolvimento Integrado (ou IDE, sigla em inglês para Integrated Development Environment), um programa de computador que reúne as características e ferramentas de apoio para a criação de aplicativos para dispositivos móveis para Android. Hoje suportando duas linguagens de programação, Java e mais recente, Kotlin, as duas voltadas para o desenvolvimento de aplicações Android.

\section{Laravel}
Gratuito, de código aberto, com suporte a recursos avançados e facilidade na construção do código, simples e legível com a utilização do padrão MVC: essas são as principais vantagens do Laravel, o que torna ele o framework preferido por muitos desenvolvedores. O Laravel se tornou o framework mais utilizado por desenvolvedores PHP que trabalham nessa ferramenta utilizando o padrão de desenvolvimento.%https://www.tecmundo.com.br/software/223718-laravel-conheca-o-framework-php-utilizado.htm

\section{MVC}

É um padrão de projeto de software que significa Model, View e Controller, fácil e eficiente, este padrão tem como pontos positivos trabalhar com reuso de código e dividir em 3 camadas a estrutura do projeto, o deixando a solução fácil e eficiente

\textbf{Model:} É a camada responsável pela regra de negócio da aplicação, é onde está as informações necessárias que toda a solução irá precisar para funcionar, desde consultas ao banco de dados, validações, notificações, entre outras coisas que serão consultadas em algum momento na aplicação.

\textbf{View:} Será onde todas essas informações e ações serão exibidas para o usuário, é na View que as informações são renderizadas, a parte interação homem-máquina fica responsável por essa camada.

\textbf{Controller:} O meio de campo das duas aplicações, o Controller fica a parte do código que gerencia o momento de chamar cada função, cada ação que deverá ser executa. O controller recebe instruções dá View, encaminha para o Model e as retorna quando necessário.

\section{Firebase}

Firebase é uma plataforma digital desenvolvida pela google utilizada para facilitar o desenvolvimento de aplicativos web ou móveis, de uma forma efetiva, rápida e simples. Suas funcionalidades principais são Firebase Authentication, que fica responsável por toda da parte de autenticação dos aplicativos, Cloud Messaging, responsável por toda parte de notificações da aplicação, podendo notificar várias plataformas, e o Realtime Messaging, onde é utilizado para mensagens instantâneas, a exemplo, o Whatsapp.

\section{PostGresSQL}

É um banco de dados objeto relacional de código aberto muito utilizando em todo o mundo, por ser seguro, robusto, de alto desempenho e multitarefa, o PostGresSQL é recomendado para projetos que precisam de alta disponibilidade, persistência num grande volume de dados, além de ter uma ferramenta de fácil aprendizado.
%https://pt.education-wiki.com/5154595-what-is-postgresql



\section{Heroku}

O Heroku é uma plataforma como serviços (PaaS) utilizada para facilitar o deploy de aplicações backend, testes em produção ou hospedagem de sites ou APIs. Bastante utilizada, o Heroku tem integrações com o GitHub, facilitando ainda mais a vida de seus usuários, suas atualizações de código são automaticamente replicadas para a aplicação na medida que o seu repositório é alterado no GitHub, além de suportar várias ferramentas PostGres e soluções backend como NodeJS, Laravel.
\end{comment}