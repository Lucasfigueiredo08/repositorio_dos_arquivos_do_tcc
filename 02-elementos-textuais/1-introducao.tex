%
% Documento: Introdução
%

\chapter{Introdução}\label{chap:introducao}  

%Define-se mobilidade como aquilo que tem “facilidade de se movimentar, andar, dançar.” \cite{mobilidade}, característica do que é móvel ou obedece às leis do movimento. 


%Mobilidade Urbana já é um termo que não encontramos no dicionário, porém de fácil compreensão,
%pois logo relacionamos a algo que de fato é ou se assemelha, basicamente a condição
%de se deslocar dentro de uma cidade, “campus”, bairro. %com objetivo de criarmos relações sociais, — como ir ao supermercado —, ou relações econômicas — como ir ao trabalho —, utilizando meios de transporte como carros, ônibus, metrô, etc. 
A mobilidade é definida como "a facilidade de se mover, andar, dançar"  \cite{mobilidade}, característica daquilo que é móvel ou obedece às leis do movimento. A mobilidade urbana já é um termo que não encontramos no dicionário, mas é fácil de entender porque rapidamente nos referimos a algo que realmente é ou se assemelha, a condição de se deslocar dentro de uma cidade, um “campus”, um bairro.

%No Brasil, com a Política Nacional de Mobilidade Urbana -- PNMU aprovada em 2012, obriga estados e municípios com mais de 20 mil habitantes a terem um planejamento de expansão pensando em como as pessoas vão se locomover, considerando o crescimento urbano e populacional  \cite{lei12587}.
No Brasil, a Política Nacional de Mobilidade Urbana (PNMU), aprovada em 2012, obriga estados e municípios com mais de 20.000 habitantes a criar um plano de expansão que leve em conta a circulação de pessoas, levando em consideração o crescimento urbano e populacional \cite{lei12587}.

%Segundo informe do Instituto de Pesquisa Econômica Aplicada -- IPEA, o aumento da produção automotiva no Brasil estimulou a crescente utilização de carros e motos em todo o território nacional. O fácil acesso a esses automóveis provocaram a redução da importância do transporte público na matriz modal, aumentou o tráfego e a emissão de poluentes \cite{ipea}.
Segundo relatório do Instituto de Pesquisa Econômica Aplicada - IPEA - o aumento da produção automobilística no Brasil tem incentivado o uso crescente de automóveis e motocicletas em todo o território nacional. O fácil acesso a esses veículos reduziu a importância do transporte público na matriz de transporte, aumentou o tráfego e aumentou as emissões de poluentes \cite{ipea}.
 
%As Cidades Inteligentes -- CI hoje são vistas como uma das soluções para alguns problemas urbanos. Pois buscam sempre melhorar o estilo de vida dos cidadãos, ao administrar recursos, analisar a qualidade do ar, gerenciar resíduos, controlar o trânsito, entre outros \cite{chourabi}.
As cidades inteligentes são hoje consideradas como uma das soluções para alguns problemas urbanos. Isso porque elas sempre tentam melhorar o estilo de vida dos cidadãos gerenciando recursos, analisando a qualidade do ar, gerenciando o tráfego e muito mais. 

%De acordo com \citeonline{namepardo}, o conceito de CI não chega a ser uma novidade no mundo acadêmico, o tema é amplamente discutido e ganhou uma nova dimensão quando passou a implementar Tecnologias da Informação e Comunicação -- TIC para construir infraestruturas e serviços de uma cidade.
Segundo \citeonline{namepardo}, o conceito de cidades inteligentes não é novidade no meio acadêmico. O tema é amplamente discutido e ganhou uma nova dimensão quando as tecnologias de informação e comunicação (TICs) passaram a ser utilizadas para construir as infraestruturas e serviços de uma cidade.

Tecnologias como GPS e \textit{Smartphones} são tendências nas cidades inteligentes, estes dispositivos propiciam a criação de novas soluções inteligentes. Com o recurso do GPS a capacidade de localizar ou buscar endereços nos mapas digitais facilitam bastante a Mobilidade Urbana, meio este que é utilizado por serviços como Google Maps\footnote{Disponível em: https://www.google.com.br/maps. Acesso em: 20 Jun. 2020.}.
%Tecnologias como tecnologias inteligentes e smartphones são tendências em cidades inteligentes, esses dispositivos possibilitam a criação de novas soluções inteligentes. Com a função GPS fica mais fácil encontrar ou buscar utilidades em mapas digitais, o meio utilizado como o Google Maps.

%Mobilidade Inteligente envolve acessibilidade, praticidade, soluções modernas e sustentáveis com forte suporte tecnológico para facilitar os deslocamentos, especialmente para usuários de transporte coletivo e privado.
Mobilidade inteligente significa acessibilidade, praticidade, soluções modernas e sustentáveis com forte suporte tecnológico para facilitar as viagens, principalmente para usuários de transporte público e privado.

%Na cidade de Macapá o número de ônibus que são oferecidos para a população é baixo, existe poucas rotas de ônibus e todas não atendem o todo espaço urbano da cidade \cite{sau2018}
Na cidade de Macapá, o número de ônibus oferecidos à população é baixo, existem poucas linhas de ônibus e todas não atendem toda a área urbana da cidade \cite{sau2018}

%Com tantos problemas causados pelo inchaço populacionl que vive os centros urbanos, cerca de 80\% da população vive na cidade de Macapá e sofre com uma estrutura de transpote %infraestrutura 
%sem preparo para comportar um número grande de pessoas \cite{tostes}.
Diante dos muitos problemas causados pelo crescimento populacional nos centros urbanos, cerca de 80\% da população vive na cidade de Macapá e sofre com uma estrutura de transporte que não está preparada para acomodar um grande número de pessoas.

\section {Problema}
\begin{comment}
	Na cidade de Macapá e Santana, as duas maiores cidade do Amapá, o número de transporte coletivo é baixo,
	existe apenas duas rodovias que interligam as cidades, 
	se locomover se torna difícil \cite{sau2018}. % 
	
	Com tantos problemas aparentes causadas pelo grande inchaço populacional que  vive nas áreas urbanas, cerca de 80\% da população das duas cidades reside em áreas urbanas e sofre com a insuficiência do sistema de transporte público oferecido \cite{tostes}.
\end{comment}

%Os acadêmicos da Universidade Federal do Amapá - Unifap têm como transporte público os ônibus e táxis, e no caso da cidade de Santana, os alunos têm apenas uma linha de ônibus que faz a rota intermunicipal \cite{sau2018}.
Os acadêmicos da Universidade Federal do Amapá - Unifap possuem ônibus e táxis como transporte público. Na cidade de Santana, há apenas uma linha de ônibus que faz o trajeto intermunicipal para os alunos \cite{sau2018}.

%No período de 2010 a 2017, a cidade de Macapá teve um aumento total de 13 ônibus ativos, e nesse mesmo período houve um aumento de 57 967 pessoas que utilizam o transporte coletivo diariamente \cite{sau2018}. %dados da Companhia de Trânsito do Amapá -- CTMAC \footnote{text}. 
No período de 2010 a 2017, a cidade de Macapá teve um aumento total de 13 ônibus ativos e no mesmo período houve um aumento de 57.967 pessoas utilizando o transporte público diariamente \cite{sau2018}.

%Cabe ressaltar que muitas rotas não assistem todas as áreas da cidade sendo necessário trocar de ônibus ou meio de transporte, o que aumenta o tempo de uso do coletivo para chegar ou sair da universidade. \cite{galiano}.
Vale ressaltar que muitas rotas não atendem todas as áreas da cidade e é necessário trocar de ônibus ou meio de transporte, o que aumenta o tempo de uso do coletivo para ir ou voltar da faculdade.

%\citeonline{sau2018} afirma que Macapá não possui outras opções de transporte coletivo além de ônibus, táxis e mototáxis, apenas lotações (corridas ilegais) e aplicativos de transporte de passageiros.
\citeonline{sau2018} conta que em Macapá, além de ônibus, táxis e mototáxis, não há outros meios de transporte público, apenas lotações (transporte ilegal) e aplicativos de transporte de passageiros.

%Questões relacionadas a Mobilidade Urbana foi levantado no trabalho, como ``Quais os principais problemas enfrentados com o transporte que utiliza?'' ou ``Você se sente satisfeito com o transporte que utiliza?'', estás perguntas serão respondidas ao decorrer da pesquisa.
O trabalho levantou questões sobre mobilidade urbana, como "Quais são os maiores problemas com o meio de transporte que você usa?" ou "Você está satisfeito com o meio de transporte que utiliza?", essas questões serão respondidas ao longo da pesquisa.






\section{Justificativa}



%Considerando o que já foi apresentado como problema sobre a Mobilidade da cidade de Macapá, o trabalho buscou uma solução existente simples, prática e que pudesse ser testada, analisada e validada pela comunidade acadêmica. 
Considerando o que já foi apresentado como um problema de mobilidade na cidade de Macapá, o trabalho buscou uma solução simples, prática e existente que pudesse ser testada, analisada e validada pela comunidade acadêmica.


%Esta solução de Mobilidade Inteligente não só busca ser uma alternativa entre ônibus, táxis, mototáxis, ser mais prático e consumir menos tempo no trajeto, mas também tem como iniciativa nossa incluir o espirito colaborativo entre os integrantes da comunidade por intermédio da oferta de carona.
Essa solução de mobilidade inteligente não pretende apenas ser uma alternativa aos ônibus, táxis e mototáxis para ser mais prática e consumir menos tempo na estrada, mas também nossa iniciativa de incorporar o espírito colaborativo entre os membros da comunidade oferecendo caronas.

%\mnote{Argumentar mais na justificativa, falar mais sobre a solução que escolhemos, além de aspectos financeiros, podemos dizer que escolhemos essa soluçãi entre as outras vistas porque nela é possível alterar o código, alterar informações no app conforme nossa demanada e necessidade e nela podemos restringir os usuários que acessam. 
%Acrescentar a importancia da mobilidade inteligente nas soluções de problemas urbanos,



\begin{comment}
que os problemas de mobilidade existentes
atualmente em Macapá atingem diretamente a comunidade acadêmica da Universidade Federal do Amapá, este trabalho buscará entender o perfil e as principais dificuldades da comunidade acadêmica, e, a partir de um estudo de soluções existentes no contexto de CI para propor uma solução que seja viável e admissível para a comunidade acadêmica da Unifap

%E para somar
%Além disso, os altos índices de criminalidade 
%afligem a população que precisa do transporte coletivo, principalmente à noite em paradas escuras
%, realidade da parada que se localiza em frente à Universidade. Muitos alunos sofrem da falta de ônibus ou de apenas uma empresa cobrir a área onde o estudante mora, e ainda existe na cidade muitas áreas sem uma empresa de ônibus cobrindo a área.  [foto ou materia que fale dessa situação]
%	\mnote{Patrícia: infelizmente não temos como afirmar isso sem uma referência e dados concretos. Melhor retirar esse parágrafo.}
%
\end{comment}

\section {Objetivo Geral}

%O projeto teve como principal foco a implementação de uma solução de mobilidade inteligente para a universidade, além de elaborar questionários que foram respondidos por integrantes da comunidade acadêmica com fins para o próprio entendimento do trabalho. 
Este trabalho teve como objetivo implementar uma solução de mobilidade inteligente existente que sirva como alternativa de entrada e saída dos acadêmicos da Unifap -- Campus Macapá.

\section{Objetivos Específicos}

Para alcançar o objetivo geral, definimos os seguintes objetivos específicos:

\begin{enumerate}

\item Realizar um levantamento de soluções de Mobilidade Inteligente;

\item Identificar o perfil da comunidade acadêmica da Unifap por meio do questionário; %\mnote{Como assim perfil? Argumente mais sobre..}

\item Definir os requisitos da solução mais viável para a comunidade acadêmica da Unifap;

\item Validar a solução com a comunidade acadêmica;

\item Avaliar a solução de mobilidade por intermédio de um questionário aplicado à comunidade;

\end{enumerate}

\section{Metodologia}
%A metodologia foi composta de quatro fases, onde a primeira fase foi realizada uma pesquisa na qual o objetivo foi encontrar soluções de Mobilidade Inteligente. Após, buscamos pela solução que apresentasse condições para que pudéssemos realizar testes e que tivesse características de serviços de caronas, transporte e compartilhamento de viagens.%\mnote{Aqui podes dizer também que nessa busca era importante encontrarmos algo que pudessemos testar no cenário da unifap, ou que tivessemos alguma liberdade para mudanças. Levando em condideração que transportes privados gerariam custos ao passaageiro e o objetivo aqui é justamente a colaboração por meio de caronas. } 
A metodologia consistiu em quatro fases, sendo a primeira fase uma pesquisa focada na busca de soluções para a mobilidade inteligente. Em seguida, procuramos uma solução que tivesse condições de realizar testes e tivesse as características de serviços de carona, transporte e compartilhamento de viagens.

%A segunda fase é constituída da aplicação de um questionário para analisar o interesse em relação a proposta e o perfil da comunidade acadêmica, buscando entender quais as necessidades existentes relacionadas a Mobilidade Urbana do Campus Macapá.
A segunda fase consistiu na aplicação de um questionário para analisar o interesse pela proposta e o perfil da comunidade acadêmica para entender as necessidades existentes relacionadas à mobilidade urbana do Campus Macapá.

%A terceira fase realizou-se mediante revisão teórica sobre o que fundamenta soluções de mobilidade, qual é a sua área de estudo e o que trazem de benefícios no ambiente de Mobilidade Urbana.
A terceira fase foi realizada por meio de uma revisão teórica sobre o que fundamenta as soluções de mobilidade, qual a sua área de pesquisa e quais os benefícios que elas trazem no ambiente da mobilidade urbana.

\begin{comment}

%segunda fase
A metodologia foi composta de quatro fases, onde a primeira é constituída da aplicação de um questionário para analisar o interesse em relação a proposta e o perfil da comunidade acadêmica, buscando entender quais as necessidades existentes relacionadas a Mobilidade Urbana do Campus Macapá.

%terceira fase
A segunda fase realizou-se mediante revisão teórica sobre o que fundamenta soluções de mobilidade, qual é a sua área de estudo e o que trazem de benefícios no ambiente de Mobilidade Urbana.

%primeira fase 
Na terceira fase foi realizada uma pesquisa na qual o objetivo foi encontrar soluções de Mobilidade Inteligente. Após, buscamos pela solução que apresentasse condições para que pudéssemos realizar testes e que tivesse características de serviços de caronas, transporte e compartilhamento de viagens. 
\end{comment}

%A quarta fase foi iniciada a implementação de um serviço de computação em nuvem com uso do Heroku \footnote{Heroku - https://www.heroku.com/platform} que se enquadra na categoria de plataforma como serviço - PaaS para hospedar a API. Ajustes necessários foram feitos no código, como alterar o layout, ajustar os botões e corrigir funcionalidades como a do Chat e das informações do veículo do caronista. 
A quarta fase começou com a implementação de um serviço de computação em nuvem usando Heroku \footnote{Heroku - https://www.heroku.com/platform}, que se enquadra na categoria Platform as a Service - PaaS, para hospedar a API. Foram feitos ajustes necessários no código, como alterar o layout, customizar os botões e corrigir funcionalidades como chat e informações do veículo do motorista.

%Contextualizar os endereços da cidade de Macapá e o Campus Macapá dentro da solução inteligente, realizar mudanças em recursos já obsoletos também foram algumas das mudanças realizadas no código. \mnote{Optou-se por uma solução que podessemos realizar ajustes, realizar sua implementação e realizarmos ajustes necessários.} 
Contextualizar as mudanças na cidade de Macapá e no campus Macapá dentro da solução inteligente, bem como modificar recursos já obsoletos, foram algumas das ações realizadas no código.

%E por fim, foi dirigido um questionário para  validação do uso da solução por alguns participantes da pesquisa, que responderam perguntas em escalas likert e subjetivas, visando a validação do mesmo, assim como, possíveis contribuições. 
E por fim, uma solicitação de validação do uso da solução foi feita por alguns participantes da pesquisa que responderam questões de escala likert e subjetiva, o que garante a validação da mesma, assim como possíveis contribuições. Foi realizada a escrita do documento de TCC de forma concomitante as fases mencionadas acima.