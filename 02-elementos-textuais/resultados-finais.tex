\chapter{Considerações Finais}
\label{chap:Considerações Finais} 
1. Retome o tema e ressalte a contribuição acadêmica do trabalho

%Este trabalho pretendeu entender sobre o mobilidade inteligente dentro das cidades e campus inteligentes para buscarmos alternativas de transporte dentre as já existentes que pudessem ser mais prático e consumir menos tempo no trajeto dos integrantes da comunidade acadêmica, além de utilizar a cultura do compartilhamento de caronas como fonte para esta solução.
Este trabalho teve como objetivo compreender a mobilidade inteligente em cidades e campi inteligentes a fim de encontrar uma solução mais prática e que economize tempo para membros da comunidade acadêmica dentre as alternativas de transporte existentes, utilizando a cultura do carpooling como fonte para esta solução.

2. Analise o cumprimento dos objetivos do seu trabalho

Para se atingir uma compreensão do objetivo geral de implementar uma solução de mobilidade existente foi necessário definir cinco objetivos específicos. O primeiro objetivo onde realizamos um levantamento das soluções de Mobilidade Inteligente. Das soluções encontradas, nem todas atendiam as características de uma solução para uma universidade, onde fosse possível restringir seu acesso para determinado grupo e que houvesse a possibilidade de integrá-la a uma base de dados restrita.

Depois, identificamos o perfil da comunidade acadêmica por meio de questionário aplicado em 2019. A análise permitiu concluir que havia um interesse por parte dos entrevistados e que seria possível implementar a solução tendo com ponto importante a possibilidade de futuramente ser integrada com as informações de acesso do SIGAA.

3. Faça apresentação e fechamento dos resultados

4. Verifique sua hipótese

5. Responda ao problema da pesquisa

6. Avalie os instrumentos de coleta de dados

7. Proponha melhorias e direcionamentos para pesquisas futuras

Modelo de considerações finais do TCC
Primeiro, você começa com a visão geral do tema do seu trabalho:
Esse trabalho pretendeu entender [apresente o tema] para [justificativa do trabalho], a partir de [metodologia utilizada].

Em seguida, descreva os objetivos específicos e apresente os resultados principais da análise:
Para se atingir uma compreensão do [objetivo geral], definiu-se três objetivos específicos. O primeiro [objetivo específico]. Verificou-se que [resultado]. Depois, [objetivo específico 2]. A análise permitiu concluir que [resultado].

Depois, verifique a sua hipótese:
Com isso, a hipótese do trabalho de que [hipótese] se [ confirmou ou se refutou], por [motivos].

E responda ao seu problema de pesquisa:
Sendo assim, [resposta ao problema de pesquisa].

Depois faça a análise dos instrumentos de dados:
Os instrumentos de coleta dos danos permitiram [avaliação dos instrumentos].

E, por fim, proponha melhorias e direcionamentos para pesquisas futuras:
Em pesquisas futuras, pode-se [melhorias e direcionamentos].

Checklist para as considerações finais do TCC
A gente quer te ajudar a fazer as considerações finais do seu trabalho. Esse checklist pode te ajudar a retomar os pontos que apresentamos até agora:

Tema e relevância do trabalho.
Cumprimento dos objetivos.
Resultados obtidos.
Verificação da hipótese.
Resposta ao problema de pesquisa.
Avaliação de instrumentos de coleta de dados. 
Melhorias e direcionamentos futuros. 