\chapter{Considerações Finais}
\label{chap:Resultados Finais} 
https://blog.mettzer.com/consideracoes-finais-tcc/

Modelo de considerações finais do TCC
Primeiro, você começa com a visão geral do tema do seu trabalho:
Esse trabalho pretendeu entender [apresente o tema] para [justificativa do trabalho], a partir de [metodologia utilizada].

Em seguida, descreva os objetivos específicos e apresente os resultados principais da análise:
Para se atingir uma compreensão do [objetivo geral], definiu-se três objetivos específicos. O primeiro [objetivo específico]. Verificou-se que [resultado]. Depois, [objetivo específico 2]. A análise permitiu concluir que [resultado].

Depois, verifique a sua hipótese:
Com isso, a hipótese do trabalho de que [hipótese] se [ confirmou ou se refutou], por [motivos].

E responda ao seu problema de pesquisa:
Sendo assim, [resposta ao problema de pesquisa].

Depois faça a análise dos instrumentos de dados:
Os instrumentos de coleta dos danos permitiram [avaliação dos instrumentos].

E, por fim, proponha melhorias e direcionamentos para pesquisas futuras:
Em pesquisas futuras, pode-se [melhorias e direcionamentos].

Checklist para as considerações finais do TCC
A gente quer te ajudar a fazer as considerações finais do seu trabalho. Esse checklist pode te ajudar a retomar os pontos que apresentamos até agora:

Tema e relevância do trabalho.
Cumprimento dos objetivos.
Resultados obtidos.
Verificação da hipótese.
Resposta ao problema de pesquisa.
Avaliação de instrumentos de coleta de dados. 
Melhorias e direcionamentos futuros. 